\newcommand{\specname}{Chaz Miller}
\newcommand{\status}{beta}
\newcommand{\ecn}{NA}
\newcommand{\revdate}{2024-02-12}
\newcommand{\rev}{000}

\newcommand{\degree}{\ensuremath{^\circ}} % degree symbol
\newcommand{\RF}{Rayleigh-Fraunhofer} %laziness

\documentclass[dvips,12pt]{article}
\renewcommand{\contentsname}{3. Index}  % I don't like 'table of contents'

\usepackage{amsmath}
%\usepackage{program}
\usepackage{a4,color,graphics,palatino,fancyhdr}
\usepackage{lastpage}  % for the END OF DATA thing
\usepackage{fancyhdr}
\usepackage{changepage}% http://ctan.org/pkg/changepage
\usepackage{graphicx}
\usepackage{fixltx2e} % allow \textsubscript

\setlength{\headheight}{15pt} %?

\setcounter{secnumdepth}{1} %toc formatting options
\setcounter{tocdepth}{1}

\lhead{\specname}
\rhead{rev.\  \rev} 
\chead{\revdate}
\cfoot{\footnotesize Page\ \thepage\ of \pageref{LastPage}}
\rfoot{\footnotesize Copyright Chaz Miller}
\pagestyle{fancy}

\title{avr555+: a general purpose firmware for AVR microcontrollers}

\author{Chaz Miller}

\begin{document}

\frenchspacing
\maketitle
\tableofcontents

%\renewcommand{\arraystretch}{1}
%\section{References}
%\begin{adjustwidth}{2.5em}{0pt}
%\begin{tabular}{   l }
%\emph {none}\\
%\end{tabular}
%\end{adjustwidth}

\section{Definitions}
\begin{adjustwidth}{2.5em}{0pt}
\begin{description}
    \item[P] Linear dimension of final print 
    \item[W] Angle-of-view factor 
    \item[AOV] Angle of view
\end{description}
\end{adjustwidth}

\section{Introduction}

\subsection{Concept}
avr555+ is a firmware for AVR microcontrollers that allows a variety of common timing, sensing, and control functions to be implemented using only passive components and wiring changes, with no need for a PC, code changes, or serial-port programmer.

AVR microcontrollers flashed with this firmware can function as timers, thermostats, motor controllers, simple human interface devices, and more, without the need for any conventional programming changes. Devices pre-flashed with avr555+ firmware can simply be "programmed" for the desired function at the time of use.

avr555+ is written for ATMEGA328 microcontrollers in common DIP packages. It uses the internal oscillator to eliminate the need for a crystal oscillator, and many functions assume 5V power supply.

Here are some of the applications that can be done with avr555+. All functions can be implemented without reprogramming:
\begin{itemize}
\item Simple oscillator -- outputs a square wave at 50\% duty cycle. The frequency can be chosen by choice of resistor, and by extension, frequency can be adjusted with a potentiometer. In this case the AVR replaces a timer such as a 555 timer.\\
\item Simple comparator -- compares voltages and toggles an output based on comparison. In this case the AVR replaces a comparator, such as LM311P.
\item Thermostat -- Similar to the comparator, but the reference voltage (which varies based on transducer used), voltage hysteresis, and "compressor delay" function can be programmed by resistors for a variety of common temperature transducers such as PT100, thermistors, and LM335.
\item Debouncer -- filter digital inputs to remove noise, using Schmitt-trigger-like filtering in software.
\item Bistable memory -- output switches between on or off when triggered
\item ADC -- read an analog voltage and output a digital value by direct binary, serial port, or PWM.
\item Stepper motor driver -- Drive a bipolor stepper motor using step-and-direction inputs, using the microcontroller to do the step translation
\item Stepper motor speed controller -- Control a bipolor stepper motor continuously using direction-and-speed inputs.
\item PWM controller -- output a variable duty cycle pulse-width modulation. The frequency of the PWM can be configured by wiring. This mode can also be "stacked" with the thermostat function, allowing output power to be adjusted separately.

\end{itemize}

\section{Programming}

\begin{tabular}{|l|l|l|c|}
\hline
Pin voltage & Top, Bottom resistors & Function & Description\\
\hline
0V& 0, 0 &Default  & Analog comparator\\
\hline
50mV& 100k, 1k& Default  & Thermostat \\
\hline
\end {tabular}

\subsection{oscillator}


\vspace{2cm}
\centering
\subsection {Program 0)
This program provides 3 simultaneous PWM controllers at a variety of
frequencies. To use
this program, connect pin 23 (PC0) to ground. To use any of the
variable output functions, connect resistors or potentiometers to pin
24 (PC1) and/or pin 25 (PC2). 

Functions:

1kHz oscillator on Timer0A: output on pin 12 (PD6). No other connections needed. 
1kHz PWM controller on Timer0B: connect pin 11 (PD5). Duty cycle will be varied according to 
voltage applied on pin 24 (PC1). To turn off output, connect pin 10 (PB7)
to ground. 

10kHz oscillator on Timer2A: connect pin 12 (PB3). No other connections
needed.
10kHz PWM controller on Timer2B: connect pin 5 (PD3). Duty cycle will be varied
according to voltage applied on pin 25 (PC2). To turn off output, connect
pin 4 (PD2) to ground. 

Variable frequency generator: Connect pin 15 (PB1) to Vcc to enable output.
Connect output on pin 16 (PB2). Frequency will be
varied according to voltage applied on pin 26 (PC3). Duty cycle will be
varied according to voltage applied on pin 27 (PC4). To turn off output,
connect pin 15 (PB1) to ground. 

Thermostat: connect as above, but instead of pulling up pin 15, connect
output of a comparator, such as the output of an LM317 or thermistor
comparator circuit. The output on pin 16 will toggle according to the
temperature. This mode works with the other PWM outputs as well by toggling
the output enable pins. 

Thermometer: 

\appendix

\end{document}

